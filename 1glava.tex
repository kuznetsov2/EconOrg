\section{Теоретические аспекты экономической эффективности инвестиционной деятельности предприятия}
Инвестиции --- довольно новый для российской экономики термин. В отечественной экономической науке понятие инвестиций употребляется как синоним капитальных вложений --- затрат материальных, трудовых и денежных ресурсов, направленных на воспроизводство основных фондов.

\subsection{Понятие инвестиций и их экономическая сущность}
Проблема повышения эффективности деятельности любого предприятия неразрывно связана с эффективным вложением капитала с целью его приумножения, или с инвестированием.

Термин <<инвестиции>> происходит от латинского \textit{invest}, что означает вкладывать. Рассмотрим определения инвестиций, которые предлагают законодательные акты.

Закон об инвестиционной деятельности \cite{1488} дает следующее определение:
\begin{quote}
	<<Инвестициями являются денежные средства, целевые банковские вклады, паи, акции и другие ценные бумаги, технологии, машины, оборудование, лицензии, в том числе и на товарные знаки, кредиты, любое другое имущество или имущественные права, интеллектуальные ценности, вкладываемые в объекты предпринимательской и других видов деятельности в целях получения прибыли (дохода) и достижения положительного социального эффекта>>.
\end{quote}

Федеральный закон <<Об инвестиционной деятельности в РФ, осуществляемой в форме капитальных вложений>> \cite{39-fz} дает схожее определение:
\begin{quote}
	<<Инвестиции --- денежные средства, ценные бумаги, иное имущество, в том числе имущественные права, иные права, имеющие денежную оценку, вкладываемые в объекты предпринимательской и (или) иной деятельности в целях получения прибыли и (или) достижения иного полезного эффекта>>.
\end{quote}

Из этих определений видно следующее. Источником прироста капитала и движущим мотивом осуществления инвестиций выступает прибыль. Инвестиции осуществляются с целью получения дохода и становятся бесполезными, если они не приносят доход. С другой стороны --- для получения требуемого дохода необходимо вложение ресурсов \cite[с. 13]{kasyanenko}.

В экономической литературе под <<инвестициями>> могут пониматься:
\begin{enumerate}
	\item [---] вложения финансовых и материально-технических средств, как в пределах Российской Федерации, так и за рубежом в целях получения экономического, экологического и социального эффектов;
	\item [---] экономические ресурсы, направляемые на увеличение реального капитала общества, т.е. на расширение и модернизацию производственного аппарата;
	\item [---] долгосрочные вложения частного или государственного капитала, имущественных или интеллектуальных ценностей в различные отрасли национальной (внутренние инвестиции) либо зарубежной (заграничные инвестиции) экономики с целью получения прибыли;
	\item [---] долгосрочные вложения средств в различные отрасли экономики с целью получения прибыли \cite[с. 9]{askinaji}
\end{enumerate}

Следует отметить, что прибыль, полученная в результате инвестирования, должна быть достаточной, чтобы возместить инвестору отказ от потребления имеющихся средств в текущем периоде, компенсировать возможный риск и потери от инфляции в будущих периодах.

Инвестирование может осуществляться как в форме вложения средств только в основной капитал (капитальные вложения), так и в нематериальные и финансовые активы.

Коротко рассмотрим другие термины, связанные с понятием <<инвестиции>>.

Инвестиционная деятельность --- процесс вложения денежных средств в объекты предпринимательской деятельности и осуществление практических действий для получения прибыли или иного полезного эффекта.

Инвестиционный цикл --- движение вложенного капитала в рамках конкретной организации от момента получения средств до момента их возмещения.

Инвестиционный процесс --- совокупность процедур формирования денежных ресурсов, их вложения в инвестиционные объекты, получение дохода и реинвестирования его части для расширения бизнеса \cite[с. 80-83]{leontev}.

\subsection{Понятие инвестиционного проекта}
Процессу инвестирования как правило предшествует разработка инвестиционного проекта. Инвестпроект позволяет оценить потребность в инвестициях, их распределение во времени, спрогнозировать доходность предполагаемых инвестиций, оценить риски инвестирования. На основании данной информации принимается решение об инвестировании \cite[с. 82]{borisova}.

Федеральный закон <<Об инвестиционной деятельности в РФ, осуществляемой в форме капитальных вложений>> \cite{39-fz} дает следующее определение.
\begin{quote}
	Под инвестиционным проектом понимается обоснование экономической целесообразности, объема и сроков осуществления капитальных вложений, в том числе необходимая проектно-сметная документация, разработанная в соответствии с законодательством РФ и утвержденными в установленном порядке стандартами (нормами и правилами), а также описание практических действий по осуществлению инвестиций (бизнес-план).
\end{quote}


В данном случае речь идет о проектах, направленных на создание реальных физических объектов (заводов, жилых домов и т.д.). Если же организации необходимо купить пакет акций другой компании, такая сделка также считается инвестпроектом.

Далее рассмотрим классификацию инвестиционных проектов по различным признакам.

По отношению друг к другу инвестпроекты бывают:
\begin{enumerate}
	\item [---] независимые (допускают одновременное и раздельное осуществление; характеристики реализации не влияют друг на друга);
	\item [---] альтернативные (не допускают одновременной реализации);
	\item [---] взаимодополняющие (реализация может происходить только совместно).
\end{enumerate}

По срокам реализации:
\begin{enumerate}
	\item [---] краткосрочные --- до трех лет;
	\item [---] среднесрочные --- от трех до пяти лет;
	\item [---] долгосрочные --- свыше пяти лет.
\end{enumerate}

По масштабам (величине капитальных вложений):
\begin{enumerate}
	\item [---] малые;
	\item [---] средние;
	\item [---] крупные;
	\item [---] мегапроекты.
\end{enumerate}

В России параметры для отнесения проектов к этим категориям не установлены.

По направленности:
\begin{enumerate}
	\item [---] коммерческие (главная цель --- получение прибыли);
	\item [---] социальные (цель --- решение определенных социальных проблем --- развитие спорта, образования, медицины и т.д.);
	\item [---] экологические (целью является решение экологических проблем, улучшение экологии и т.д.).
\end{enumerate}

Эта классификация носит чисто условный характер. Большинство коммерческих проектов приносит также социальную выгоду.


В зависимости от величины риска:
\begin{enumerate}
	\item [---] надежные (безрисковые) проекты --- вероятность получения гарантированных результатов высокая;
	\item [---] рисковые проекты --- характеризуются высокой степенью неопределенности затрат и результатов.
\end{enumerate}

Любой инвестиционный проект состоит из трех основных элементов (рисунок \ref{fig:element}).

\begin{figure}[!ht]
	\centering
	\includegraphics[width=0.7\linewidth]{element}
	\caption{Основные элементы инвестиционного проекта}
	\label{fig:element}
\end{figure}

Жизненным циклом инвестиционного проекта считается временной интервал между моментом появления и окончания реализации проекта. Окончанием обычно является ввод в действие объектов, достижение проектом заданных результатов, прекращение финансирования проекта, вывод проекта из эксплуатации.

На рисунке \ref{fig:prcycle} показаны три фазы жизненного цикла инвестпроекта: предынвестиционная, инвестиционная и эксплуатационная.

\begin{figure}[!hb]
	\centering
	\includegraphics[width=0.85\linewidth]{prcycle}
	\caption{Жизненный цикл инвестиционного проекта}
	\label{fig:prcycle}
\end{figure}

Фаза 1 --- предынвестиционная, подготовительная стадия, предшествует основному объему инвестирования. Сюда входит юридическое оформление инвестпроекта, поиск источников финансирования и др.

Фаза 2 --- инвестиционная, включает инвестирование и осуществление проекта.

Фаза 3 --- эксплуатационная (производственная), начинается после ввода объекта в эксплуатацию и является самой продолжительной.

Наиболее значимыми качественными параметрами инвестпроекта являются срок строительства ($\text{Т}_{\text{стр}}$) и срок окупаемости ($\text{Т}_{\text{ок}}$).

Срок строительства связан с рациональным использованием ресурсов в период 1 и 2 фазы. Срок окупаемости же зависит от всех трех стадий инвестпроекта, но такие параметры проекта, как качество и эффективность доказываются на эксплуатационной стадии \cite[с. 162--166]{sergeev}




































