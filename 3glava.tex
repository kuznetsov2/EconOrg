\section{Индивидуальное задание}
\subsection{Проект расширения ассортимента продукции}
В данной главе предлагается рассмотреть инвестиционный проект, а также оценить его экономическую эффективность.

Основным видом деятельности предприятия является производство молока. Объем производимой продукции составляет в среднем 10 тонн молока в день. Из них 1 тонна молока в день продается в розничных магазинах под собственным брендом, 9 тонн сдается на молокозаводы для последующей переработки. %За счет внедрения цеха предприятие сможет расширить ассортимент продукции, тем самым повысить выручку и доход. 
Предприятие имеет налаженные каналы реализации продукции под собственным брендом. 

Для повышения эффективности производства и реализации предприятия и повышения его доходов, предлагается расширить ассортимент производимой продукции за счет производства продуктов переработки молока и расширения рынков сбыта за счет новой продукции. Для этого предлагается открытие нового производства --- цеха по переработке молока.

Целью инвестпроекта является организация цеха по переработке молока. В результате реализации проекта ассортимент производимой продукции расширится. 

Предлагается приобрести модульный цех по переработке молока компании «Колакс». Для монтажа цеха необходима подготовленная площадка размером 15х8, подвод водопровода, канализации, а также электроэнергии.

Стоимость цеха составляет 8 млн. руб., стоимость монтажа 800 тыс.руб. Монтаж производится силами поставщика оборудования. Также в эту сумму входит подвод водопровода и канализации, электроэнергии. Обучение работников, а также необходимые расходные материалы предоставляет поставщик оборудования.

Общая стоимость проекта 8800 тыс. руб., в том числе 8800 тыс. руб. капитальные вложения. Финансирование проекта предлагается осуществлять по следующей схеме: собственные средства: 5000 тыс. руб. (57\%), заемные средства: 3800 тыс. руб.

%В таблице \ref{my-label} приведены необходимые расходы на введение цеха в эксплуатацию.

Размер необходимого кредита составляет 3,8 млн руб. Процентная ставка по кредиту принималась 22\% годовых.


Продукция, которая будет производиться:
\begin{itemize}
	\item молоко пастеризованное; 
	\item напиток кисломолочный кефирный; 
	\item сметана или сливки питьевые; 
	\item творог; 
	\item сыр мягкий Адыгейский; 
	\item масло сливочное Крестьянское; 
	\item сыворотка.
\end{itemize} 

Этапы проекта:\\
1.	покупка и монтаж линии по производству молочной продукции, \\
2.	производство молочной продукции, \\
3.	расширение системы сбыта произведенной продукции. 

Для реализации проекта необходимо собрать команду проекта: руководитель, технолог, рабочие.

Расчет экономической эффективности и технико-экономические показатели приведены в таблице \ref{my-label}

% Please add the following required packages to your document preamble:
% \usepackage{multirow}
\begin{table}[]
	\small
	\centering
	\caption{Технико-экономические показатели}
	\label{my-label}
	\setlength{\extrarowheight}{0.7mm}
	\begin{tabularx}{\textwidth}{|p{8.05cm}|p{8.05cm}|}
		\hline
		\multirow{6}{8cm}{Ассортимент и количество продукции, получаемой за сутки:} & молоко пастеризованное, фасованное в полиэтиленовые пакеты;     \\ %\cline{2-2} 
																				& напиток кисломолочный кефирный, фасованный в полиэтиленовые пакеты; \\ %\cline{2-2} 
																				& сметана или сливки питьевые, фасованные в пластиковые стаканчики;   \\ %\cline{2-2} 
																				& творог, весовой;                                                    \\ %\cline{2-2} 
																				& сыр мягкий Адыгейский, фасованный в пищевую пленку;                 \\ %\cline{2-2} 
																				& масло сливочное Крестьянское, весовое.                              \\ \hline
		Объем перерабатываемого молока:                                         & 3 000 кг/сутки                                                      \\ \hline
		Количество обслуживающего персонала:                                    & 4 человека                                                          \\ \hline
		Энергопотребление цеха в сутки:                                         & 163 кВт/ч                                                           \\ \hline
		\multicolumn{2}{|l|}{Стоимость электроэнергии:}                                                                                               \\ \hline
		5 руб./кВт·ч х 163 кВт·ч х 0,5 = 407,5                                  & 408 руб./сутки                                                      \\ \hline
		Продолжительность месяца:                                               & 30 суток                                                            \\ \hline
		Зарплата одного работника:                                              & 800 руб./сутки                                                      \\ \hline
		Стоимость расходных материалов:                                         & 6000 руб./сутки                                                     \\ \hline
		Стоимость цеха с учётом НДС:                                            & 8 000 000 руб.                                                      \\ \hline
		Стоимость монтажных работ:                                              & 800 000 руб.                                                        \\ \hline
		Себестоимость молока:                                                   & 20 руб./кг.                                                         \\ \hline
		\multicolumn{2}{|l|}{Цена реализации готовых продуктов:}                                                                                      \\ \hline
		молоко;                                                                 & 34 руб./кг.                                                         \\ \hline
		напиток кефирный                                                        & 35 руб./кг.                                                         \\ \hline
		творог                                                                  & 150 руб./кг.                                                        \\ \hline
		сыр Адыгейский                                                          & 220 руб./кг.                                                        \\ \hline
		масло Крестьянское                                                      & 330 руб./кг.                                                        \\ \hline
		сметана                                                                 & 180 руб./кг.                                                        \\ \hline
		Расчет эффективности производства:                                      & Расчет дохода на 1 сутки:                                           \\ \hline
		Расчет затрат на 1 сутки:                                               & стоимость реализации готовой продукции                              \\ \hline
		стоимость молока: 3 000 кг. х 20 руб./кг =  60 000 руб.                 & молоко пастеризованное — 1294 кг х 34 руб/кг = 43 996 руб.          \\ \hline
		стоимость электроэнергии: 408 руб.                                      & напиток кисломолочный кефирный — 500 кг х 35 руб/кг = 17 500 руб.   \\ \hline
		зарплата работающих: 3 х 800 руб + 1200 руб = 3 600 руб.                & сметана или сливки питьевые — 136 кг х 180 руб/кг = 24 480 руб.     \\ \hline
		стоимость расходных материалов: 6 000 руб.                              & творог — 76 кг х 150 руб/кг = 11 400 руб.                           \\ \hline
																				& сыр мягкий Адыгейский — 50 кг х 220 руб/кг = 11 000 руб.            \\ \cline{2-2}
																				& масло сливочное Крестьянское — 19 кг х 330 руб/кг = 6 270 руб.      \\ \hline
		Всего:  70 008 руб.                                                     & Всего: 114 646 руб.                                                 \\ \hline
	\end{tabularx}
\end{table}

Далее проведем анализ эффективности инвестиционного проекта с помощью статических методов оценки.



По формуле \eqref{eq:srok_okup} вычислим срок окупаемости инвестпроекта:
\begin{equation*}
T=\dfrac{IC}{P} = \dfrac{8 800 000}{946 139} = 9,3\  \text{мес.}
\end{equation*}

Полученное значение позволяет 

По формуле \eqref{eq:rentab} вычислим рентабельность инвестпроекта:
\begin{equation*}
R = \dfrac{P}{IC} = \dfrac{10407533}{8800000} = 1,18
\end{equation*}

Данный показатель для первого года . После выплаты кредита доходы увеличатся, соответственно рентабельность проекта повысится.

Посчитаем среднюю рентабельность проекта. Для расчета используем показатели за два года. По формуле \eqref{sred_rentab}:
\begin{equation*}
R_{\text{ср}} = \dfrac{P}{K_{\text{ср}}} \cdot 100\% = \dfrac{12836725}{8800000} \cdot 100\% = 145,87\%
\end{equation*}

Полученные показатели 



Далее проведем анализ с применением динамических методов оценки эффективности.

Найдем чистую приведенную стоимость инвестпроекта. Затраты по проекту осуществляются единовременно, поэтому для расчета воспользуемся формулой \eqref{npv}. Так как часть средств для финансирования проекта привлекается в банке под высокий процент, ставку дисконтирования примем равной 25\%.
\begin{equation*}\label{npv_mod}
NPV =  \dfrac{10407533}{1+0,25} + \dfrac{15265917}{(1+0,25)^2}- 8800000 = 9296213
\end{equation*}

Значение чистого дисконтированного дохода положительно, следовательно проект может считаться эффективным.

Рассчитаем индекс доходности $PI$:
\begin{equation*}
PI = \dfrac{\dfrac{10407533}{1+0,25} + \dfrac{15265917}{(1+0,25)^2}}{8800000} = 2,06
\end{equation*}

Индекс доходности равен 2,06 --- значение больше единицы, показывает что проект является экономически привлекательным.

Полученные показатели позволяют сделать вывод, что инвестпроект является эффективным.

Чтобы определить пороговое значение рентабельности, найдем внутреннюю норму доходности инвестпроекта. Данный расчет произведем при помощи электронных таблиц. Для вычисления воспользуемся встроенной формулой IRR или ВСД. В качестве аргумента функции возьмем входящий денежный поток проекта за 24 месяца. Полученный показатель $IRR = 39,59\%$ позволяет сделать следующий вывод: если стоимость капитала достигнет значения 39,59\%, то данный проект станет непривлекательным.

Для расчета чистой приведенной стоимости была взята ставка дисконтирования 25\%. Для более точной оценки внутреннюю норму доходности необходимо сравнить со стоимостью капитала. Для проекта были использованы собственные средства фирмы, а также привлеченные средства. Рассчитаем средневзвешенную стоимость капитала ($WACC$) и сравним полученное значение с внутренней нормой доходности. Доходность собственных средств примем равной 20\%.
\begin{equation*}
WACC = 0,57 \times 20\% + 0,43 \times 22\% = 20,86\%
\end{equation*}

Полученные значения  $IRR = 39,59\% > WACC= 20,86\%$, следовательно проект считается рентабельным.



