В данной главе предлагается инвестиционный проект в виде капитальных вложений, а также оценка его экономической эффективности.

Целью инвестпроекта является организация цеха по переработке молока. В результате реализации проекта ассортимент производимой продукции расширится. 

Предприятие занимается производством молока. В данный момент 1 тонна в день продается в розничных магазинах под собственным брендом. 10 тонн сдается на молокозаводы. За счет внедрения цеха предприятие сможет расширить ассортимент продукции, тем самым повысить выручку и доход. 

Предприятие имеет налаженные каналы реализации продукции под собственным брендом. 


Предлагается приобрести модульный цех по переработке молока компании Х. Для монтажа цеха необходима подготовленная площадка размером 15х8, подвод водопровода, канализации, а также электроэнергии.

Стоимость цеха составляет 8 млн. руб., стоимость монтажа 800 тыс.руб. Монтаж производится силами поставщика оборудования. Также в эту сумму входит подвод водопровода и канализации, электроэнергии. Обучение работников, а также необходимые расходные материалы предоставляет поставщик оборудования.

В таблице приведены необходимые расходы на введение цеха в эксплуатацию.
Предлагается финансировать проект по следующей схеме: 5 млн за счет средств предприятия и 3,8 млн за счет кредита взятого в банке под 22%.


























