\subsection*{Введение}
\addcontentsline{toc}{section}{Введение}
Инвестиции --- довольно новый для российской экономики термин. В отечественной экономической науке понятие инвестиций употребляется как синоним капитальных вложений --- затрат материальных, трудовых и денежных ресурсов, направленных на воспроизводство основных фондов.

Инвестиционная деятельность --- это процесс вложения средств с последующим получением дохода или иного социального эффекта. В настоящее время инвестиции играют ключевую роль в развитии экономики. За счет инвестиций происходит модернизация производства, создание новых продуктов и услуг, повышение производительности труда и качества производимой продукции.

Рассмотрение любого инвестиционного проекта требует предварительного анализа и оценки. Особенно сложной и трудоемкой является оценка инвестиционных проектов в реальные инвестиции. Инвестиционная привлекательность проекта требует детального анализа множества показателей и принятия правильного решения под влиянием риска, неопределенности, инфляции.

Актуальность темы обусловлена необходимостью научного обоснования процесса инвестирования и управления инвестиционной деятельностью на предприятии.

Целью работы является изучение теоретических аспектов экономической эффективности инвестиционной деятельности и применение изученных методов на примере инвестиционного проекта.

Для достижения поставленной цели необходимо решить следующие задачи:\\
1. исследовать теоретические аспекты инвестиционного проектирования;\\
2. провести анализ финансового состояния предприятия;\\
3. разработать инвестиционный проект и дать оценку экономической эффективности проекта.

Объектом исследования выступает сельскохозяйственное предприятие ООО «Агрофирма Острожка», основным видом деятельности которого является разведение молочного крупного рогатого скота, производство сырого молока.

Предметом исследования является оценка экономической эффективности инвестиционного проекта по расширению ассортимента производимой продукции на предприятии ООО «Агрофирма Острожка».

В ходе работы были использованы следующие методы: сравнение, анализ, горизонтальный и вертикальный анализ финансовой отчетности, коэффициентный анализ, анализ научной литературы и периодических источников.

Теоретической основой работы выступили труды отечественных ученых В. М. Аскинадзи, Т. Г. Касьяненко, В. Е. Леонтьева, Т. В. Погодиной и др.

Нормативной базой работы являются законодательные акты и нормативные документы РФ.

Информационной базой является бухгалтерская отчетность ООО «Агрофирма Острожка» за 2014--2016 г.г.

В первой главе рассматривается сущность понятия инвестиций, формирования инвестиционного проекта и расчета показателей эффективности.

Во второй главе дается организационно-экономическая характеристика предприятия, проводится анализ основных финансовых показателей деятельности предприятия, анализ финансовой устойчивости, деловой активности и рентабельности предприятия.

Третья глава посвящена разработке и оценке экономической эффективности инвестиционного проекта дальнейшего развития ООО «Агрофирма Острожка».
