\subsection{Оценка эффективности инвестиционного проекта}


Эффективность инвестиционной деятельности представляет собой соотношение результатов и затрат и позволяет дать оценку целесообразности реализации инвестпроекта.

Существует два подхода к оценке эффективности инвестиционного проекта. Первый подход заключается в том, что фактор времени не учитывается. Второй же подход, напротив, учитывает фактор времени на показатели оценки эффективности инвестиционной деятельности.

Методика оценки эффективности инвестиционных проектов является многоуровневой и включает несколько этапов:
\begin{enumerate}
 	\item [1)] сравнение показателя общей рентабельности проекта со средним банковским процентом по депозитам в целях определения наиболее оптимального вложения;
	\item [2)] сравнение рентабельности инвестиционных проектов со средним темпом инфляции в стране в целях минимизации денежных потерь от инфляции;
	\item [3)] сравнение проектов по объему необходимых инвестиций в целях обеспечения минимальных потребностей в кредитах и займах;
	\item [4)] сравнение инвестиционных проектов по срокам окупаемости с целью наиболее быстрой окупаемости проекта;
	\item [5)] оценка стабильности поступлений от реализации проекта с целью обеспечения поступления регулярных доходов;
	\item [6)] сравнение общей рентабельности инвестиционных проектов без учета фактора времени с целью обеспечения максимальной рентабельности вложений;
	\item [7)] сравнение общей рентабельности инвестиционных проектов с учетом временного фактора с целью исследования эффективности инвестиционных проектов с позиций сегодняшнего временного периода и определения текущей стоимости.
\end{enumerate}

Принятие решения о реализации инвестиционного проекта --- процесс многогранный и многоуровневый, определяющий его абсолютный эффект, относительную и сравнительную эффективность с учетом инфляции и альтернативных вариантов инвестирования.

Существуют разные методы оценки привлекательности инвестпроектов. Наиболее популярные --- это срок окупаемости инвестиций, рентабельность капитальных вложений, средняя рентабельность за период жизни инвестпроекта, индекс доходности, чистый дисконтированный доход, внутренняя форма доходности.

Далее рассмотрим методы оценки эффективности инвестпроекта, не учитывающие влияние фактора времени на показатели эффективности. Такие методы также называют статическими.

1. Срок окупаемости инвестиций ($T$) --- простейший метод оценки привлекательности инвестиционного проекта. Рассчитывается как отношение суммы капитальных вложений ($IC$) к размеру годовой прибыли ($P$):
\begin{equation}\label{eq:srok_okup}
T=\dfrac{IC}{P}.
\end{equation}

Данный метод показывает число лет, необходимое, чтобы полностью окупить вложенные средства. Недостаток заключается в том, что не учитывается временной фактор. В случаях, когда ожидается получение прибыли в разные временные промежутки, рекомендуется модернизировать формулу --- осуществлять дисконтирование полученных в будущем выгод:
\begin{equation}\label{eq:srok_okup_disk}
T = \dfrac{IC}{P'},
\end{equation}
где $P'$ --- приведенная (дисконтированная) к единому временному периоду среднегодовая прибыль.

2. Рентабельность капитальных вложений ($R$) --- показатель эффективности инвестиций, обратный сроку окупаемости:
\begin{equation}\label{eq:rentab}
R = \dfrac{P}{IC}.
\end{equation}

Имеет те же недостатки, что и предыдущий. Для денежных потоков, осуществляемых в различное время рекомендуется использовать модернизированную формулу, учитывающую 
временной фактор.

3. Средняя рентабельность проекта ($R_{\text{ср}}$) --- отношение среднегодовой прибыли к средней за период величине капитальных вложений ($K_{\text{ср}}$), выраженное в процентах:
\begin{equation}\label{sred_rentab}
R_{\text{ср}} = \dfrac{P}{K_{\text{ср}}} \cdot 100\%.
\end{equation}

Данный метод оценки в определенной мере учитывает продолжительность жизненного цикла инвестиционного проекта.

Приведенные методы позволяют получить поверхностную оценку эффективности инвестиционной деятельности. Для получения более точной оценки необходимо также использовать динамические методы оценки эффективности.

1. Чистая дисконтированная стоимость ($NPV$) представляет собой разность между совокупным доходом ($CF_i$) и всеми понесенными затратами ($IC$) за весь период жизненного цикла инвестпроекта с учетом фактора времени:
\begin{equation}\label{npv}
NPV = \sum \dfrac{CF_i}{1+r} - IC.
\end{equation}

Данная формула используется, когда существуют единовременные затраты по инвестпроекту. Если необходимы разновременные затраты --- применяется модифицированная формула:
\begin{equation}\label{npv_mod}
NPV = \sum \dfrac{CF_i}{(1+r)^i} - \sum \dfrac{IC_i}{(1+r)^i},
\end{equation}
где $IC_i$ --- затраты по инвестпроекту в $i$-м году; $r$ --- ставка дисконтирования.

Инвестиционный проект может считаться эффективным, если значение чистого дисконтированного дохода имеет положительное значение. Если инвестпроект будет осуществлен при отрицательном значении показателя, то инвестор понесет убытки, т.е. проект неэффективен.

2. Индекс доходности ($PI$) представляет собой отношение суммы приведенных выгод ($CF$) к величине капиталовложений ($IC$):
\begin{equation}\label{ind_dohod}
PI = \dfrac{\sum\dfrac{CF_i}{(1+r)^i}}{\sum\dfrac{IC_i}{(1+r)^i}}.
\end{equation}

Инвестпроект считается экономически привлекательным, если значение индекса доходности больше единицы, в противном случае он неэффективен. Если показатель индекса доходности равен единице, то проект нуждается в дополнительном обосновании, так как в этом случае после реализации инвестпроекта ценность компании не изменится.

3. Внутренняя норма доходности ($IRR$) определяет пороговое значение рентабельности, при котором обеспечивается равенство нулю чистого дохода за весь период жизни инвестпроекта. Таким образом, приведенная стоимость будущих денежных поступлений равна приведенной стоимости оттоков денежных средств с учетом возмещения инвестированного капитала.

Внутренняя норма рентабельности рассчитывается следующим образом:
\begin{equation}\label{vnutr_norma_rent}
IRR = r_1 + \dfrac{(r_2 - r_1) \cdot NPV(r_1)}{NPV(r_1) - NPV(r_2)},
\end{equation}
где $r_1$ --- первая ставка дисконтирования, обеспечивающая положительное значение $NPV$; $r_2$ --- вторая ставка дисконтирования, обеспечивающая отрицательное значение $NPV$.

($IRR$) --- такая ставка дисконта, которая обеспечивает равенство современной стоимости доходов по инвестпроекту и затрат на его осуществление. На практике ($IRR$) обычно находят методом подбора различных пороговых значений рентабельности. Инвестиционный проект считается рентабельным, если внутренняя норма доходности с учетом риска превышает среднюю стоимость капитала в данном секторе экономики, в противном случае проект считается нецелесообразным.

Существует следующая зависимость между чистой дисконтированной стоимостью, индексом доходности и внутренней нормой рентабельности:
\begin{itemize}
	\item [---] если $NPV > 0$, то $PI > 1$ и $IRR > r$;
	\item [---] если $NPV = 0$, то $PI = 1$ и $IRR = r$;
	\item [---] если $NPV < 0$, то $PI < 1$ и $IRR < r$.
\end{itemize}

Если чистая дисконтированная стоимость проекта имеет положительное значение, то индекс доходности имеет значение больше единицы и показатель внутренней нормы рентабельности выше стоимости капитала.

Как правило, ни один из приведенных показателей не является достаточным для принятия решения об осуществлении инвестиций. Данное решение принимается на основе использования всех вышеперечисленных методов оценки и с учетом интересов всех участников инвестиционного проекта \cite[с. 49--56]{pogodina}.




































































