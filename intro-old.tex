Для общества развитие инвестиционного проектирования позволяет формировать новые предприятия, что создает новые рабочие места. Разработка новых предприятий обеспечивает повышение уровень налогов, поступаемых в бюджет, а значит, увеличивает объемы финансирования социальных проектов. 

Для организации роль инвестиционного проектирования заключается в следующем. В настоящий момент для развития деятельности каждый предприниматель должен уделять существенное внимание направлениям вложения инвестиций. Для этого оптимальным вариантом является формирование и разработка инвестиционного проекта. Осуществление вложения средств на основании разработанного плана обеспечивает высокую вероятность получения прибыль и успешного осуществления деятельности предприятия. Хорошо продуманный инвестиционный проект - это залог высоких конечных финансовых результатов работы предприятия. 

Инвестиционное проектирование позволяет рассчитать предварительные затраты, единовременные, постоянные и переменные расходы и соотнести их с планируемыми показателями прибыли. Инвестиционный проект также предполагает детальное изучение рынка, что также немаловажно для формирования прибыльности деятельности предприятия. Предварительный расчет и составление проекта позволяет избежать риска банкротства данного направления. Так если расчетная рентабельность инвестиционного проекта ниже среднего процента банковского кредита, то проект должен быть отклонен. 

Итак, хорошо продуманный инвестиционный проект позволяет проанализировать сложившуюся ситуацию, управлять имеющимися ресурсами компании и обеспечить стабильное наращивание выручки и прибыли, а также деловой активности и устойчивости предприятия на рынке.

Объект исследования --- ООО <<Пермснаб>>.

Предмет исследования --- экономическая эффективность инвестиционной деятельности предприятия ООО <<Пермснаб>>.

Целью работы является анализ экономической эффективности инвестиционной деятельности предприятия ООО <<Пермснаб>> и разработка мероприятий по совершенствованию работы предприятия.

Для реализации цели в работе были поставлены следующие задачи:
\begin{itemize}
	\item изучить теоретические основы формирования инвестиционного проекта;
	\item проанализировать финансово-хозяйственную деятельность предприятия и ее эффективность;
	\item разработать инвестиционный проект и оценить его эффективность.
\end{itemize}

В ходе работы были использованы следующие методы: сравнение, анализ, горизонтальный и вертикальный анализ финансовой отчетности, коэффициентный анализ, анализ научной литературы и периодических источников, регрессионный анализ.

Теоретической и методологической основой исследования явились положения и методики фундаментальных и прикладных наук в области экономики, управления персоналом, менеджмента и маркетинга.

Достоверность исследования основывается на применении отечественных и международных законодательных актов, нормативных документов РФ, данных Федеральной службы государственной статистики РФ, данных международных исследовательских центров, научных публикаций в открытых федеральных и международных изданиях. В работе были использованы исследования и разработки следующих отечественных и зарубежных учёных: Аникина Б.А., Бауэрсокса Д. Дж, Гаджинского А.М., Ламберта Д.М., Левинсона У., Лукинского В.С. и других авторов.


В первой главе рассматривается сущность формирования инвестиционного проекта и расчета показателей эффективности. Во второй главе дается организационно-экономическая характеристика предприятия, проводится анализ основных финансовых показателей деятельности предприятия, анализ финансовой устойчивости, деловой активности и рентабельности предприятия. Третья глава посвящена разработке инвестиционного проекта дальнейшего развития ООО «Пермснаб».
