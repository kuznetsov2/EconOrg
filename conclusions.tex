\subsection*{Выводы и предложения}
\addcontentsline{toc}{section}{Выводы и предложения}

Инвестиционная деятельность является безусловно одной из важнейших вех в развитии любой организации. Необходимость инвестиций обусловлена такими причинами как обновление существующей материально-технической базы, увеличение объемов выпуска продукции, повышение качества производимых товаров и услуг, освоение новых видов деятельности.

Наиболее важным этапом в процессе управления инвестициями выступает оценка экономической эффективности инвестиционного проекта. От качества выполненной оценки зависит правильность принятия решения об инвестировании. Основным направлением анализа является определение показателей возможной экономической эффективности инвестиций --- отдачи от вложений.

Объектом исследования выступило предприятие ООО <<Агрофирма Острожка>>. В исследуемом периоде в целом предприятие показывало нестабильную динамику: выручка имеет сильную зависимость от объема реализованной продукции. Себестоимость продаж также показывает нестабильность. В 2016 году валовая прибыль значительно повысилась --- это свидетельствует о повышении эффективности управления предприятием. Рентабельность продаж в 2016 году достигла уровня 10\%.

Анализ финансовой устойчивости предприятия показал неустойчивое финансовое положение на протяжении всего исследуемого периода. Основные проблемы --- это зависимость от заемных источников финансирования.

Проанализировав бухгалтерскую отчетность предприятия можно сделать вывод, что в исследуемом периоде ООО <<Агрофирма Острожка>> развивается: показатели рентабельности показывают положительную динамику.

В третьей главе курсовой работы был предложен инвестиционный проект и проведена оценка его экономической эффективности. Для повышения эффективности деятельности предприятия и повышения прибыли было предложено организовать цех по переработке молока. В результате реализации проекта расширится ассортимент производимой продукции и повысятся доходы предприятия.

Общая стоимость проекта 8,8 млн. рублей. Финансирование проекта было предложено производить по следующей схеме: 5 млн. рублей за счет собственных средств предприятия и 3,8 млн. рублей за счет взятого в банке кредита под 22\% годовых.

Был проведен анализ инвестпроекта с помощью статических и динамических методов оценки экономической эффективности.

В результате проведения анализа экономической эффективности инвестпроекта с помощью статических методов оценки были получены следующие показатели. Срок окупаемости проекта составил 9,3 месяца, рентабельность составила 18\%. Средняя рентабельность за два года составила 45\%. Учитывая данные показатели можно сделать вывод, что инвестиционный проект является привлекательным.

Так как полученные в результате анализа экономической эффективности инвестпроекта с помощью статических методов оценки показатели оказались удовлетворительными, инвестпроект был также проанализирован с помощью динамических методов.

Были рассчитаны следующие показатели: чистая приведенная стоимость, индекс доходности, внутренняя норма доходности. При расчетах ставка дисконтирования принималась равной 25\%. С учетом этой ставки была рассчитана средневзвешенная стоимость капитала, которая составила 20,86\%.

Чистая приведенная стоимость инвестпроекта составила 9296213 рублей. Так как значение положительно, проект считаем эффективным.

Полученное значение индекса доходности 2,06 больше единицы, что показывает, что проект является экономически привлекательным.

Таким образом, на основании рассчитанных показателей можно заключить, что инвестиционный проект является эффективным.

Для определения порогового значения рентабельности была рассчитана внутренняя норма доходности инвестпроекта. Полученный показатель 39,59\% позволяет сделать вывод, что пока стоимость капитала ниже этого значения, проект является привлекательным.

По итогам проведенного анализа с помощью статических и динамических методов оценки экономической эффективности инвестиционного проекта, можно заключить, что предложенный проект организации цеха по переработке молока является эффективным.

Следует учитывать, что предложенный инвестиционный проект и проведенный анализ его экономической эффективности преследуют учебные цели и выполнены исходя из темы курсовой работы. Это означает, что его применение в реальной хозяйственной деятельности требует доработки.

В работе не были учтены такие факторы, как сезонное изменение цен на сырье для производства, изменение цен на реализуемую продукцию, изменение конъюнктуры рынка, изменение экономической ситуации в стране. Данный проект также не учитывает налоговые платежи предприятия, транспортные издержки на доставку готовой продукции до места реализации, доставку сырья до места производства и другие возможные издержки.

В настоящее время на рынке существует большое разнообразие кредитных и лизинговых продуктов. Поэтому рекомендуется рассмотреть разные варианты финансирования проекта и провести их сравнительный анализ.


















